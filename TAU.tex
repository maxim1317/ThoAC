
% \documentclass[14pt,final,openany]{extreport}      % Sets 12pt font, equation numbers on right
% \usepackage{amsmath,amssymb,amsfonts,amstext} % Typical maths resource packages
% \usepackage{graphics}                 % Packages to allow inclusion of graphics
% \usepackage{color}                    % For creating coloured text and background

% \usepackage[utf8]{inputenc}
% \usepackage[english,russian]{babel}
% \usepackage{vmargin}
% \setpapersize{A4}
% \setmarginsrb{2cm}{1.5cm}{1cm}{1.5cm}{0pt}{0mm}{0pt}{13mm}
% \usepackage{indentfirst}
% \sloppy

\documentclass[a4paper,12pt]{extreport}
 
\usepackage{extsizes}
% \usepackage{cmap} % для кодировки шрифтов в pdf
\usepackage[T1,T2A]{fontenc}
\usepackage[utf8]{inputenc}
\usepackage[russian]{babel}
% \usepackage{misccorr}
\renewcommand{\rmdefault}{cmr}
\renewcommand{\sfdefault}{cmss}
\renewcommand{\ttdefault}{cmtt}


% }
\usepackage{float}

\usepackage{graphicx} % для вставки картинок
\usepackage{amssymb,amsfonts,amsmath,amsthm} % математические дополнения от АМС
\usepackage{indentfirst} % отделять первую строку раздела абзацным отступом тоже
\usepackage[usenames,dvipsnames]{color} % названия цветов
% \usepackage{makecell}
\usepackage{multirow} % улучшенное форматирование таблиц
% \usepackage{ulem} % подчеркивания
 
\usepackage{subfiles}
\usepackage{enumitem}
\usepackage{caption}
\usepackage{placeins}

% \usepackage{cmap}              % pure T1 fonts 
% \usepackage[resetfonts]{cmap}  % pure T1 fonts, reset CM
% \usepackage{mmap}              % cmap + mathematics (ASCII)
\usepackage[noTeX]{mmap}       % cmap + mathematics (Unicode)

\usepackage{tasks}

\usepackage[]{hyperref}
\hypersetup{pdftex,colorlinks=true,allcolors=blue}
\usepackage{hypcap}
\usepackage[shortcuts]{extdash}
% \sloppy
% \def\subsubsection{\@startsection{subsubsection}{3}%
%   \z@{.5\linespacing\@plus.7\linespacing}{-.5em}%
%   {\normalfont\itshape}}

\graphicspath{{./fig/t1/}{./fig/t2/}{./fig/t3/}{./fig/t4/}{./fig/t5/}}

\linespread{1.3} % полуторный интервал


\title{Теория Автоматического Управления}
\author{Капалин Иван Владимирович}
% \institute[МИСиС ИК]{НИТУ МИСиС\\ Кафедра инженерной кибернетики}

\date{}

% \renewcommand*\contentsname{Содержание}


\theoremstyle{plain}
\newtheorem{theor}{\bf Теорема}
\newtheorem{coroll}{\bf Следствие}
\newtheorem{statement}{\bf Утверждение}
%%%%%%%%%%%%%%%%%%%%%%%%%%%%%%%%%%%%%%%%%%%%%%%%%%%%%%%
\theoremstyle{definition}
\newtheorem{defi}{\bf Определение}
% \newtheorem{task}{\bf Задача  \thetask}
\theoremstyle{remark}
\newtheorem{remark}{\bf Замечание}
\newtheorem{examp}{\bf Пример}
\theoremstyle{plain}
%%%%%%%%%%%%%%%%%%%%%%%%%%%%%%%%%%%%%%%%%%%%%%%%%%%%%%%

\newcommand{\eref}[1]{(\ref{#1})}
\newcommand{\const}{\mathop{\mathrm{const}}\nolimits}
\newcommand{\col}{\mathop{\mathrm{col}}\nolimits}
\newcommand{\diag}{\mathop{\mathrm{diag}}\nolimits}
\newcommand{\rank}{\mathop{\mathrm{rank}}\nolimits}
\newcommand{\spec}{\mathop{\mathrm{spec}}\nolimits}
\newcommand{\rang}{\mathop{\mathrm{rang}}\nolimits}
\newcommand{\rg}{\mathop{\mathrm{rg}}\nolimits}
\newcommand{\RE}{\mathop{\mathrm{Re}}\nolimits}
\newcommand{\IM}{\mathop{\mathrm{Im}}\nolimits}
\newcommand{\cnt}[1]{\overline{#1}}
\newcommand{\BF}[1]{\mathbb{#1}}
\newcommand{\IT}[1]{\mathcal{#1}}
\newcommand{\LAP}[1]{\mathop{\mathcal{L}}\nolimits\left\{#1\right\}}
\newcommand{\INVLAP}[1]{{\mathop{\mathcal{L}}\nolimits^{-1}\left\{#1\right\}}}
\newcommand*{\hm}[1]{#1\nobreak\discretionary{}%
{\hbox{$\mathsurround=0pt #1$}}{}}
\newcommand{\lse}{\leqslant}
\newcommand{\ges}{\geqslant}

\newcommand{\thickhat}[1]{\mathbf{\hat{\text{$#1$}}}}

% \usepackage{etoolbox}
% \preto\section{\filbreak}
\widowpenalties 1 10000
\raggedbottom

\frenchspacing
\righthyphenmin=2

\relpenalty   = 9999
\binoppenalty = 9999

\begin{document}
\maketitle

\tableofcontents

\chapter*{От автора}

    This is the first section.
     
    Lorem  ipsum  dolor  sit  amet,  consectetuer  adipiscing  
    elit.   Etiam  lobortisfacilisis sem.  Nullam nec mi et 
    neque pharetra sollicitudin.  Praesent imperdietmi nec ante. 
    Donec ullamcorper, felis non sodales...

\subfile{tex/chapters/t_1}
\subfile{tex/chapters/t_2}
\subfile{tex/chapters/t_3}
\subfile{tex/chapters/t_4}
\subfile{tex/chapters/t_5}

\begin{thebibliography}{0}

    \bibitem{kim:uch} Ким~Д.~П.
    \emph{Теория автоматического управления. Линейные системы.} М.: ФИЗМАТЛИТ, 2007. – 168 с., ISBN~5-9221-0379-2.

    \bibitem{kim:prac} Ким~Д.~П., Дмитриева~Н.~Д.
    \emph{Сборник задач по теории автоматического управления. Линейные системы.} М.: ФИЗМАТЛИТ, 2007. – 168 с., ISBN~968-5-9221-0873-7.

    \bibitem{voron} Воронов~А.~А.
    \emph{Теория автоматического управления. Часть 1} Н. А. Бабаков, А. А. Воронов, А. А. Воронова и др.; Под ред. А, А„ Воронова.—2-е изд., перераб. и доп. — М.: Высш. шк., 1986.

    \bibitem{voron2} Воронов~А.~А.
    \emph{Теория автоматического управления. Часть 2} Н. А. Бабаков, А. А. Воронов, А. А. Воронова и др.; Под ред. А, А„ Воронова.—2-е изд., перераб. и доп. — М.: Высш. шк., 1986.

    \bibitem{yashin} Муромцев~Д.~Ю.
    \emph{Анализ и синтез дискретных систем} Д.Ю. Муромцев, Е.Н. Яшин. – Тамбов : Изд-воФГБОУ ВПО «ТГТУ», 2011.

    \bibitem{andr} Андриевский~Б.~Р.
    \emph{Теоретические основы автоматизированного управления}: кхм

    \bibitem{pervoz} Первозванский~А.~А.
    \emph{Курс теории автоматического управления : учебное пособие для вузов} А. А. Первозванский. - Москва: Наука, 1986.

    \bibitem{pandia}
    \emph{Метод пространства состояний в теории непрерывных систем автоматического управления} \url{http://pandia.ru/text/80/295/1133.php}

\end{thebibliography}

\end{document}