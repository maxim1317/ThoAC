\documentclass[TAU.tex]{subfiles}
\begin{document}

\chapter{Основные понятия, структура и классификация систем автоматического управления}

С древних времен человек хотел использовать предметы и силы природы в своих целях, то есть управлять ими. Теория управления пытается ответить на вопрос «как нужно управлять?». До XIX века науки об управлении не существовало, хотя первые системы автоматического управления уже были (например, ветряные мельницы «научили» разворачиваться навстречу ветру). Развитие теории управления началось в период промышленной революции. Сначала это направление в науке разрабатывалось механиками для решения задач регулирования, то есть поддержания заданного значения частоты вращения, температуры, давления в технических устройствах (например, в паровых машинах). Отсюда происходит название «теория автоматического регулирования». Позднее выяснилось, что принципы управления можно успешно применять не только в технике, но и в биологии, экономике, общественных науках.

\section {Процессы управления}
%Процессы управления (слайд 3) - понятия, общая информация о процессах + таблица

Процессы управления и обработки информации в системах любой природы изучает наука кибернетика. Один из ее разделов, связанный главным образом с техническими системами, называется теорией автоматического управления. 

\begin{center}
Процессы управления\\ [4pt]
\begin{tabular}{|p{6.5cm}|p{6.5cm}|}
  % after \\: \hline or \cline{col1-col2} \cline{col3-col4} ...
  \hline
  {\Large В живой природе} & {\Large В неживой природе} \\ [4pt]
  \hline
  Естественный отбор & Наведение на цель орудия \\
  \hline
  Терморегуляция у животных & Поддержание температуры в печи\\
  \hline
  Поддержание равновесия животными & Поддержание равновесия робота\\
  \hline
  Увеличение рождаемости в стране & Поддержание скорости на моторе\\
  \hline
  Уничтожение клеток определенного типа (вирусных, инфекционных и т.п.)& Поддержание фиксированной высоты летального аппарата\\
  \hline
  Повышение работоспособности работников предприятия & Поддержание заданного напряжения\\
  \hline
\end{tabular}
\end{center}

\section{Характеристика процессов управления}
%Характеристика процессов управления (слайд 4) - пара слов о каждой характеристике + примеры

Общие характеристики всех процессов управления:
\begin{itemize}
  \item Прием информации - поиск и обнаружение сигналов (выделение сигналов из шума). Примеры: камера, глаз, датчики давления, скорости, положения и т.п., общение.
  \item Хранение информации - процесс поддержания исходной информации в виде, обеспечивающем выдачу данных по запросам конечных пользователей. Примеры: память животных, память на носителях - USB, HDD, CD, DVD.
  \item Преобразование информации - процесс изменения формы представления информации или ее содержания. Примеры: анализ рынка, те или иные вычисления.
  \item Выработка управляющего воздействия - подача напряжения на мотор, передача указаний подчиненным, поворот руля. %-не очень понятно, уточнить.
\end{itemize}

\section{Исходные положения ТАУ}
\subsubsection{САУ. Структурная схема и понятия}
%САУ. Структурная схема и понятия (слайд 5-6) - схема, описание, термины. Что входит в схему, как работает, назначение каждого элемента

\defi{\bf Управление} - целенаправленное воздействие не объект или устройство. Управление может быть автоматическим, т.е. без участия человека, ручным, т.е. в присутствии человека, или полуавтоматическим, т.е. работающим при участии человека.

\defi{Объект управления} {\it (ОУ)} - устройство, которым нужно и можно управлять. Это может быть автомобиль, самолет электродвигатель и т.д.




\end{document}