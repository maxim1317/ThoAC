\documentclass[../../TAU.tex]{subfiles}
\begin{document}
\chapter{Устойчивость непрерывных систем управления}

    \defi Система является {\it устойчивой}, если после исчезновения того или иного возмущения ее поведение вернется к заданному режиму. В противном случае система считается нейтральной. либо неустойчивой. \par
    Большинство систем являются нелинейными, а линейными являются лишь их приближения. А. М. Ляпунов в 1892 году показал, что в некоторых случаях по первому приближению можно судить об устойчивости нелинейной системы.

    \theor{\it Теоремы Ляпунова об устойчивости нелинейной системы:}
    \begin{enumerate}
        \item Если все корни характеристического уравнения линеаризованной модели являются левыми, то невозмущенное движение соответствующей нелинейной системы асимптотически устойчиво.
        \item Если среди корней характеристического уравнения линеаризованной модели имеется правый корень, то невозмущенное движение соответствующей нелинейной системы неустойчиво.
        \item Случай, когда среди корней характеристического уравнения линеаризованной модели имеются нейтральные корни (корни на мнимой оси), но нет правых корней, называют критическим. В критическом случае по линеаризованной модели нельзя судить об устойчивости невозмущенного движения нелинейной системы. 

    \end{enumerate}

\section{Устойчивость в линейных системах}

    \defi{}Система 
    \begin{equation}\label{LIN_DS}
        \dot x = Ax + bu, x(0) = x_0
    \end{equation} 
    называется 
    {\it асимптотически устойчивой по Ляпунову}, если 
    $x_\text{о}(t)\rightarrow0$ 
    при 
    $t\rightarrow\infty$ 
    и любом $x_0$.\par
    Как известно, в системах вида
    \eref{LIN_DS}
    решение можно представить в виде суммы свободного и вынужденного движений:
    $$
        x(t) = x_\text{о}(t) + x_\text{ч}(t),
    $$
    где $x_\text{ч}(t)$ --- частное решение неоднородного уравнения при 
    $x_0 = 0$, $x_\text{о}(t)$~--- общее решение уравнения при 
    $u\equiv0$.
    \par
    Общее решение $x_\text{о}(t)$ однородного уравнения описывает свободное движение системы управления (т. е. движение при отсутствии внешних воздействий), определяемое только начальными условиями. Частное решение $x_\text{ч}(t)$ описывает вынужденное движение, определяемое внешними воздействиями.

\subsection{Основное условие устойчивости}

    Характеристическое уравнение системы управления совпадает с характеристическим уравнением дифференциальных уравнений и имеет вид
    $\alpha(\lambda)=a_0\lambda^n+a_1\lambda^{n-1}+...+a_n=0$, 
    где $\alpha(s)$ --- 
    характеристический полином. Если 
    $\lambda_i, (i=1,2,...,q)$ ---
    корни характеристического уравнения кратности
    $k_i, (k_1+k_2+...+k_q=n)$, 
    то общее решение однородного уравнения имеет вид 
    $x_\text{o}(t)=\sum_{i=1}^q{Pi(t) e^{\lambda i t}}$, 
    где 
    $P_i(t)=C_1^{(i)}+C_2^{(i)} t+...+C_{ki}^{(i)} t^{k_i-1}$; 
    где 
    $C_{ki}^{(i)}$ --- постоянные   интегрирования. 
    В частном случае, когда все корни простые, 
    $x_\text{ч}(t)=\sum_{i=1}^nCi e^{\lambda i t}$. 
    По правилу Лопиталя можно показать, что 
    $P_i(t) e^{\lambda i t}\rightarrow 0 $
    при 
    $t\rightarrow\infty$ 
    тогда и только тогда, когда действительная часть корня 
    $\lambda_i$
    отрицательна: 
    $Re\ {i}<0$. 

    \defi{\it Основное условие устойчивости.} Для того чтобы система управления была устойчива, необходимо и достаточно, чтобы все корни ее характеристического уравнения имели отрицательную вещественную часть (т.е. лежали в левой полуплоскости). 

\subsection{Необходимое условие устойчивости}

    \theor{\it Необходимое условие устойчивости} Для того чтобы система была устойчива, необходимо, чтобы все коэффициенты ее характеристического уравнения были строго одного знака: 
    $$
        a_0>0,\ a_1>0, ...\ ,\ a_n>0\ \text{или}\ a_0<0,\ a_1<0, ...\ ,\ a_n<0\
    $$
    Если одно из этих условий не выполняется,то система является неустойчивой.

    \begin{proof}
        Представим характеристический полином в виде разложения
        $\alpha(\lambda)=a_0(\lambda-\lambda_1) (\lambda-\lambda_2) ... (\lambda-\lambda_n)$.
        Действительному отрицательному корню
        $\lambda_k = -\alpha_k, (\alpha_k>0)$
        в разложении соответствует множитель
        $\lambda - \lambda_k=\lambda+\alpha_k$.
        Паре комплексно-сопряженных корней с отрицательной вещественной частью 
        $\lambda_l=-\alpha_l+j \beta_l$ 
        и 
        $\lambda_{l+1}=-\alpha_l-j \beta_l$, 
        $(\alpha_l,\ \beta_l > 0)$ 
        соответствует множитель 
        $(\lambda-\lambda_l) (\lambda-\lambda_{l+1}) = (\lambda+\alpha_1 - j \beta_l) (\lambda+\alpha_1 + j \beta_l) = (\lambda+\alpha_l)^2+\beta_l^2$.
        Следовательно если все корни характеристического уравнения имеют отрицательные вещественные части, то характеристический полином может быть представлен как произведение полиномов первой и второй степени с положительными коэффициентами, и соответственно все его коэффициенты при 
        $a_0>0$ будут положительными и при 
        $a_0<0$ --- отрицательными.
    \end{proof}

\section{Алгебраические критерии устойчивости}

    \defi{\it Алгебраическими критериями} устойчивости называются такие  условия, составленные из коэффициентов характеристического уравнения, при выполнении которых система устойчива, а при невыполнении --- неустойчива. При проведении исследования устойчивости следует первоначально проверить выполнение необходимого условия устойчивости.

\subsection{Критерий Гурвица}

    Определим матрицу Гурвица
    $$
        H =\begin{pmatrix}
            a_{n-1} & a_{n-3} & 0 & \ldots & 0 \\
            1 & a_{n-2} & a_{n-4} & \ldots & 0 \\
            \vdots & \vdots & \ddots && \vdots \\
            0 & \ldots & a_{3} & a_1 & 0 \\
            0 & \ldots & a_4 & a_2 & a_0
        \end{pmatrix}\in\BF{R}^{n\times n},
    $$
    \theor[Гурвица] Пусть 
    $a_i>0,\; i=\cnt{0,n-1}$.
    Полином $\alpha(\lambda)$ устойчив тогда и только тогда, когда все главные миноры $\Delta_i$ матрицы Гурвица положительны. \\
    Из коэффициентов характеристического полинома 
    $\alpha(\lambda)=a_0(\lambda-\lambda_1) (\lambda-\lambda_2) ... (\lambda-\lambda_n)$
    составляется определитель n-го порядка:
    $$
        \Delta_i =
        \begin{vmatrix}
            a_{n-1}& \ldots& *\\
            *& \ddots & *\\
            *& \ldots & a_{n-i}
        \end{vmatrix}
    $$

    \examp Исследовать на устойчивость полином\\
    $\alpha(\lambda) = \lambda^4+\lambda^3+5\lambda^2+10\lambda+1$ 
    с помощью критерия Гурвица.

    {\it Решение:}\par
    Необходимое условие устойчивости выполнено, составим матрицу Гурвица.
    $$
        H = 
        \begin{pmatrix}
            10 & 1 & 0 & 0\\
            1 & 5 & 1 & 0 \\
            0 & 10 & 1 & 0\\
            0 & 1 & 5 & 1 \\
        \end{pmatrix}
    $$

    Найдем определители Гурвица: 
    $$
    \Delta_1=10,\ \Delta_2 = 
    \begin{vmatrix}
        10 & 1\\
        1  & 5
    \end{vmatrix}
    =49,\ \Delta_3=
    \begin{vmatrix}
        10 & 1 & 0\\
        1  & 5 & 1\\
        0 & 10 & 1\\
    \end{vmatrix}
    =-51<0.
    $$
    Система неустойчива.

    \examp Исследовать на устойчивость полином\\
    $\alpha(\lambda) = \lambda^4+5\lambda^3+7\lambda^2+11\lambda+8$ 
    с помощью критерия Гурвица.

    {\it Решение:}\par
    Необходимое условие устойчивости выполнено, составим матрицу Гурвица.
    $$
        H = 
        \begin{pmatrix}
            11 & 5 & 0 & 0\\
            8 & 7 & 1 & 0 \\
            0 & 11 & 5 & 0\\
            0 & 8 & 7 & 1 \\
        \end{pmatrix}
    $$

    Найдем определители Гурвица: 
    $$
    \Delta_1=11,\ \Delta_2 = 
    \begin{vmatrix}
        11 & 5\\
        8  & 7
    \end{vmatrix}
    =77 - 40=37,\ \Delta_3=
    \begin{vmatrix}
        11 & 5 & 0\\
        8  & 7 & 1\\
        0 & 11 & 5\\
    \end{vmatrix}
    =64,\ \Delta_4=1 \cdot \Delta_3=64.
    $$
    Так как все определители Гурвица больше нуля, то полином устойчив. 


\end{document}