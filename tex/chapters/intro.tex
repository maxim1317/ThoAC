\documentclass[../../TAU.tex]{subfiles}
\begin{document}

    \chapter*{От автора}

        \markboth{ОТ АВТОРА}{}
        % Перед вами методическое пособие, которое было составлено на основе как классических, так и современных книг по теории автоматического управления (ТАУ).  
        % Пособие представляет собой в основном выборочные факты из ТАУ, объединенные одной идей — дать студенту математический язык, используемый в этой дисциплине, и дать представление об областях его применения.
        % Хотя в ТАУ далеко не последнюю роль играет инженерная составляющая, в данном пособии она почти не освещается в силу особенности образования автора, проходившего обучение в основном по фундаментальным дисциплинам. 
        % В этом текст вы сможете найти 5 способов описания динамических систем (обыкновенные дифференциальные уравнения, в том числе системы, передаточные функции, временные функции и структурные схемы), которые в равной степени используются в ТАУ. Также вы найдете здесь основные понятия изучаемой дисциплины (объект управления, цель управления, задающее воздействие, регулируемая величина, а также устойчивость, наблюдаемость, управляемость, стабилизатор, наблюдатель, время регулирования, перерегулирование и др.), которые используются для выражения важных теорем (критериев устойчивости Гурвица, Михайлова, Найквиста, Харитонова, теоремы о стабилизации системы). Кроме того, в работе присутствуют примеры, которые должны помочь студенту лучше понять смысл абстрактных утверждений и связи между ними. Примеры частично были составлены автором, частично студентами, без помощи которых это пособие не появилось на свет. 
        % Еще одной важной составляющей пособия является список литературы, в котором можно найти более подробное или наоборот пространное изложение материала. 
        % Я надеюсь, что все изложенное здесь даст необходимый студенту импульс для интересного знакомства с дисциплиной, а также поддержку в сдаче зачетов или экзаменов по этому курсу.
        Перед вами методическое пособие по теории автоматического управления, составленное на основе как классических, так и современных источников

        Пособие представляет собой в основном выборочные факты из ТАУ, объединенные одной идей, — дать студенту математический язык, используемый в этой дисциплине, и дать представление об областях его применения.

        В пособии вы сможете найти не только способв описания динамических систем, которые в равной степени используются в ТАУ, но и основные понятия изучаемой дисциплины, которые используются для выражения важных теорем.

        Кроме того, в работе присутствуют примеры, которые должны помочь студенту лучше понять смысл абстрактных утверждений и связи между ними. Еще одной важной составляющей пособия является список литературы, в котором можно найти более подробное и понятное студенту изложение материала.

        Я надеюсь, что изложенная в пособии информация позволит студенту не только подробно изучить все теории и аспекты изучаемого предмета, но и успешно подготовиться к сдаче зачетов или экзаменов по курсу теории автоматического управления.
\end{document}