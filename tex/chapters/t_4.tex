\documentclass[../../TAU.tex]{subfiles}
\begin{document}

\chapter{Стабилизация линейных систем}

\section{Полиномиальная стабилизация}

    Данный метод стабилизации предполагает представление непрерывной части в виде дискретной модели. 
    При этом вся система считается чисто дискретной, что позволяет применять для стабилизации регуляторов хорошо разработанные методы теории дискретных систем. 
    Регулятор представляется в виде отношения полиномов (цифровой фильтр), параметры которого предстоит найти. 
    Поиск параметров осуществляется на основе данных о желаемом расположении полюсов дискретной системы (задача размещения полюсов), которые, как известно, определяют качество переходного процесса. 
    В простых случаях, когда порядок регулятора невысок, применимы графоаналитические методы, однако более универсальными являются методы поиска решения полиноминального уравнения.
    \cite[стр. 98]{yashin}


    Пусть дана замкнутая система с регулятором $R(s) = \frac{\phi(s)}{\psi(s)}$ в отрицательной обратной связи. 
    Для решения задачи полиномиальной стабилизации воспользуемся следующей теоремой
    \begin{theor}\label{TH1} 
        Если для системы с ПФ $W(s) = \frac{\beta(s)}{\alpha(s)}$ выполнены условия
        \begin{enumerate}
            \item НОД $(\beta,\; \alpha) = 1$,
            \item $m = \deg\beta< \deg\alpha = n$,
        \end{enumerate}
        тогда для любого полинома $\gamma(s),\; \deg\gamma(s) \ge n+m-1$ существуют полиномы $\psi(s)$ и $\phi(s)$, такие что
        $\gamma(s) = \alpha(s)\psi(s) + \beta(s)\phi(s)$. При этом обеспечивается грубость замкнутой системы (не происходит сокращения степеней в хар. пол-ме).
    \end{theor}

    \begin{coroll}
        Если $\deg\gamma = 2n-1$, тогда найдутся полиномы со степенями $\deg\psi = n-1$, $\deg\phi \le n-1$ (нестрого физически реализуемый регулятор).
    \end{coroll}
    \begin{coroll}
        Если $\deg\gamma = 2n$, тогда найдутся полиномы со степенями $\deg\psi = n$, $\deg\phi \le n-1$ (строго физически реализуемый регулятора).
    \end{coroll}

\section{Управляемость и наблюдаемость \cite[стр. 269]{voron}}

    При решении задач управления методами пространства состояний предварительно рассматриваются некоторые свойства динамических систем, которые однозначно характеризуют возможности использования известной модели ДС для управления объектом. 
    Такими свойствами являются управляемость и наблюдаемость. 
    Наличие этих свойств у объектов управления позволяет синтезировать управление с помощью простых математических операций.
    \cite[стр. 62]{yashin} 

    Пусть дана система вида

    \begin{equation}\label{GEN_DS}
        \left\{
        \begin{aligned}
            \dot x &= f(t,x,u),\\
            y &= h(t,x,u),\quad x(t_0) = x_0
        \end{aligned}
        \right.
    \end{equation}
    где $x(t)$ --- состояние системы, $u(t)$ --- вход/управление, $y(t)$ --- выход, $f$ и $h$ --- некоторые функции.

    \begin{defi}
        Система \eref{GEN_DS} называется {\it управляемой}, если для любого момента $t_1$ и точки фазового пространства $x_1$ найдется управление $u(t)$, определенное на отрезке $[0,t_1]$, такое что для решения $x(t)$ системы \eref{GEN_DS} выполнено: $x(t_1) = x_1$ при условии, что $x(0)=0$.
    \end{defi}

    \begin{defi}
        Система \eref{GEN_DS} называется {\it наблюдаемой}, если при $u\equiv0$ для любого начального момента $t_0$ из условия $y(t)\equiv 0$ при $t\ge t_0$ следует $x(t_0) = 0$ и, наоборот, из $x(t_0)=0$ следует $y(t)\equiv0$ при $t\ge t_0$.
    \end{defi}

\subsection{Управляемость линейных объектов. Критерий управляемости}

    Пусть матрицы $A$ и $B$ постоянны. Введем в рассмотрение матрицу управляемости:
    \begin{equation}\label{LIN_DS}
        \left\{
        \begin{aligned}
            \dot x &= Ax + bu,\\
            x(0) &= x_0,
        \end{aligned}
        \right.
    \end{equation}
    где $A\in\BF{R}^{n\times n}$, имеет место следующая
    \begin{theor}[Критерий управляемости]
        Линейный объект $\eref{LIN_DS}$ управляем тогда и только тогда, когда выполнено условие 
        $$
            \rg K_{A,b} = n,
        $$
        где
        $$
            K_{A,b} = \left[b,\; Ab,\; \cdots,\; A^{n-1}b\right]\in\BF{R}^{n\times n}
        $$.
    \end{theor}

    \proof \cite[стр. 305]{voron2}
    \par
    \textbf{Программное управление}
    \par
    Из доказательства критерия управляемости было получено, что управление вида
    $$
        u(t) = b^Te^{A^T(t_1-t)}W^{-1}(t_1)x_1
    $$
    переводит систему за время $t_1$ в состояние $x_1$ из состояния $x_0 = 0$. Это пример {\it программного управления}, когда в системе нет обратной связи.

    Как перевести систему из произвольного состояния $x_0$ в $x_1$?

    Используя идею аналогичную идеи из доказательства, получим управление вида
    $$
        u(t) = b^Te^{A^T(t_1-t)}W^{-1}(t_1)(x_1 - e^{At_1}x_0).
    $$

\subsection{Модальное управление}

    \defi{\it Модальное управление} --- это управление, в котором цель управления достигается за счет назначения корней характеристического полинома ЗАМКНУТОЙ системы. Здесь под целью управления имеется в виду настройка начальной фазы работы САУ или, строго говоря, {\it переходного режима}. Позже будут рассмотрены различные характеристики переходного режима.

    Модальное управление {\it по состоянию} имеет вид
    $$
        u = -kx,
    $$
    где $k\in\BF{R}^{1\times n}$ --- вектор коэффициентов обратной связи. В результате замыкания получим замкнутую систему
    $$
        \left\{
        \begin{aligned}
            \dot x &= (A - bk)x,\\
            x(0) &= x_0.
        \end{aligned}
        \right.
    $$

    \begin{theor} 
        Пусть пара $\{A,\; b\}$ --- управляема и задан полином $\gamma(s)$. Тогда существует вектор 
        $k\in\BF{R}^{1\times n}$, 
        такой что 
        $\sigma(A-bk) = \left\{s_1,\; s_2,\;\ldots,\; s_n\right\}$, где $s_i$ --- корни $\gamma(s)$.
    \end{theor}

    \textbf{Алгоритм построения вектора $k$}

    \begin{enumerate}
    \item 
        Вычислить коэффициенты полинома $\gamma(s) = \prod_{i=1}^{n}(s-s_i)$;
    \item 
        Вычислить $K_{A,b}$ ($|K_{A,b}|\neq0$, если система управляема);
    \item 
        Найти $\chi_A(s) = s^n + a_{n-1}s^{n-1} + \ldots + a_1 s + a_0$;
    \item 
        Вычислить $K_{\hat A,\; \hat b}$, $\{\hat A,\; \hat b\}$ --- каноническая форма управляемости;
    \item 
        Перемножить $M = K_{\hat A, \hat b} K^{-1}_{A,b}$;
    \item 
        Вычесть $\hat k_i = \gamma_{i-1} - a_{i-1}, i=\cnt{1,n}$;
    \item 
        Умножить: $k = \hat k M$;
    \item 
        Записать $u = -k x$.
    \end{enumerate}

\end{document}