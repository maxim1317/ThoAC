\documentclass[../../TAU.tex]{subfiles}
\begin{document}

\chapter{Стабилизация линейных систем}

\section{Полиномиальная стабилизация}

    Данный метод стабилизации предполагает представление непрерывной части в виде дискретной модели. 
    При этом вся система считается чисто дискретной, что позволяет применять для стабилизации регуляторов хорошо разработанные методы теории дискретных систем. 
    Регулятор представляется в виде отношения полиномов (цифровой фильтр), параметры которого предстоит найти. 
    Поиск параметров осуществляется на основе данных о желаемом расположении полюсов дискретной системы (задача размещения полюсов), которые, как известно, определяют качество переходного процесса. 
    В простых случаях, когда порядок регулятора невысок, применимы графоаналитические методы, однако более универсальными являются методы поиска решения полиноминального уравнения.
    \cite[стр. 98]{yashin}


    Пусть дана замкнутая система с регулятором $R(s) = \frac{\phi(s)}{\psi(s)}$ в отрицательной обратной связи. 
    Для решения задачи полиномиальной стабилизации воспользуемся следующей теоремой
    \begin{theor}\label{TH1} 
        Если для системы с ПФ $W(s) = \frac{\beta(s)}{\alpha(s)}$ выполнены условия
        \begin{enumerate}
            \item НОД $(\beta,\; \alpha) = 1$,
            \item $m = \deg\beta< \deg\alpha = n$,
        \end{enumerate}
        тогда для любого полинома $\gamma(s),\; \deg\gamma(s) \ge n+m-1$ существуют полиномы $\psi(s)$ и $\phi(s)$, такие что
        $\gamma(s) = \alpha(s)\psi(s) + \beta(s)\phi(s)$. При этом обеспечивается грубость замкнутой системы (не происходит сокращения степеней в хар. пол-ме).
    \end{theor}

    \begin{coroll}
        Если $\deg\gamma = 2n-1$, тогда найдутся полиномы со степенями $\deg\psi = n-1$, $\deg\phi \le n-1$ (нестрого физически реализуемый регулятор).
    \end{coroll}
    \begin{coroll}
        Если $\deg\gamma = 2n$, тогда найдутся полиномы со степенями $\deg\psi = n$, $\deg\phi \le n-1$ (строго физически реализуемый регулятора).
    \end{coroll}

    \examp
    {
        Дана система с ПФ $W(s) = \frac{s-1}{s^2+3s-2}$. Синтезировать (построить) регулятор методом полиномиальной стабилизации с характеристическим полиномом замкнутой системы $\gamma(s) = s^3+2s^2+s+1$.
    }

    \textit{Решение}
    {
        Пусть $W(s) = {\beta(s)\over \alpha(s)}$, тогда
        $$
            m=deg\ \beta =1 \ \ \ n=deg\ \alpha=2
        $$
        так как $m<n$, то для полинома $\gamma(s)$,
        $$
            deg\ \gamma(s) \ges n + m - 1
        $$
        существуют полиномы $\psi(s)$ и $\varphi(s)$, что $\gamma(s)=\alpha(s)\cdot\psi(s)+\beta(s)\cdot\varphi(s)$.
        т.к. $deg\ \gamma(s)=2n-1$, то регулятор является нестрого физически реализуемым (следствие 2) и
        $$
            deg\ \psi=n-1,\ \ \  deg\ \varphi \lse n-1
        $$,
        Положим, что $\varphi(s)=\varphi_1\cdot s+\varphi$, $\psi(s)=s+\psi_0$ и 
        $R(s)={\varphi(s)\over\psi(s)}$
        Тогда
        $$
            (s^2+3s-2)(s+\psi_0)+(s-1)(\varphi_1s+\varphi_0)=s^3+2s^2+s+1,\ \forall s
        $$ 
        $$
            s^3+(3+\psi_0+\varphi_1)s^2+(3\psi_0-2-\varphi_1+\varphi_0)s+(-2\psi_0-\varphi_0)=s^3+2s^2+s+1
        $$
        $$
        \begin{cases}
            3+\psi_0+\varphi_1=2 \\
            3\psi_0-2-\varphi_1+\varphi_0=1 \\
            -2\psi_0-\varphi_0=1
        \end{cases}
        $$
        Отсюда
        $$
            \psi_0={3\over2} \ \ \ \varphi_1= -{5\over2} \ \ \  \varphi_0=-4
        $$
        то есть
        $$
            R(s)=\frac{5s+8}{3-2s}
        $$
    }

    \examp
    {
        Дана система с ПФ $W(s) = \frac{s}{s^2+s-1}$. Синтезировать (построить) регулятор методом полиномиальной стабилизации с характеристическим полиномом замкнутой системы $\gamma(s) = s^4+3s^3+4s^2+3s+1$.
    }

    \textit{Решение}
    {
        $$
            W(s)=\frac{\beta(s)}{\alpha(s)}
        $$
        $$
            deg\ \beta(s)=m=1;\ \ \ deg\ \alpha(s)=n=2
        $$
        так как $m<n$, тогда для полинома $\gamma(s)$
        $$
            deg\ \gamma(s)\lse n+m
        $$
        существуют полиномы $\psi(s)$ и $\varphi(s)$, что $\gamma(s)=\alpha(s)\cdot\psi(s)+\beta(s)\cdot\varphi(s)$.
        Положим, что $R(s)=\frac{\varphi(s)}{\psi(s)}$.


        Так как $deg\ \gamma(s)=2n$, то регулятор является строго физически реализуемым и $deg\ \psi(s)=n=2$, $deg\ \varphi\lse n-2\Rightarrow deg\ \varphi\lse n-1$.

        Положим, что $\varphi(s)=\varphi_1\cdot s+\varphi$, $\psi(s)=s+\psi_0$ и 
        $R(s)={\varphi(s)\over\psi(s)}$
        Тогда

        $$
            (s^2+s-1)(\psi_1\cdot s^2+\psi_2\cdot s+\psi_0)+s\cdot(\varphi_1\cdot s+\varphi_0)=s^4+3s^5+4s^2+3s+1
        $$
        $$
        \begin{cases}
            \psi_1=1\\
            \psi_2+\psi_1=3\\
            \psi_0+\psi_2-\psi_1+\varphi_1=4\\
            \psi_0-\psi_2+\varphi_0=3\\
            -\psi_0=1
        \end{cases}
        $$
        Отсюда получим
        $$
        \psi_0=-1\ \ \psi_1=1\ \ \psi_2=2\ \ \varphi_0=6\ \ \varphi_1=4
        $$
        то есть
        $$
            R(s)=\frac{4s+6}{s^2+2s-1}
        $$
    }



\section{Управляемость и наблюдаемость \cite[стр. 269]{voron}}

    При решении задач управления методами пространства состояний предварительно рассматриваются некоторые свойства динамических систем, которые однозначно характеризуют возможности использования известной модели ДС для управления объектом. 
    Такими свойствами являются управляемость и наблюдаемость. 
    Наличие этих свойств у объектов управления позволяет синтезировать управление с помощью простых математических операций.
    \cite[стр. 62]{yashin} 

    Пусть дана система вида

    \begin{equation}\label{GEN_DS}
        \left\{
        \begin{aligned}
            \dot x &= f(t,x,u),\\
            y &= h(t,x,u),\quad x(t_0) = x_0
        \end{aligned}
        \right.
    \end{equation}
    где $x(t)$ --- состояние системы, $u(t)$ --- вход/управление, $y(t)$ --- выход, $f$ и $h$ --- некоторые функции.

    \begin{defi}
        Система \eref{GEN_DS} называется {\it управляемой}, если для любого момента $t_1$ и точки фазового пространства $x_1$ найдется управление $u(t)$, определенное на отрезке $[0,t_1]$, такое что для решения $x(t)$ системы \eref{GEN_DS} выполнено: $x(t_1) = x_1$ при условии, что $x(0)=0$.
    \end{defi}

    \begin{defi}
        Система \eref{GEN_DS} называется {\it наблюдаемой}, если при $u\equiv0$ для любого начального момента $t_0$ из условия $y(t)\equiv 0$ при $t\ge t_0$ следует $x(t_0) = 0$ и, наоборот, из $x(t_0)=0$ следует $y(t)\equiv0$ при $t\ge t_0$.
    \end{defi}

\subsection{Управляемость линейных объектов. Критерий управляемости}

    Пусть матрицы $A$ и $B$ постоянны. Введем в рассмотрение матрицу управляемости:
    \begin{equation}\label{LIN_DS}
        \left\{
        \begin{aligned}
            \dot x &= Ax + bu,\\
            x(0) &= x_0,
        \end{aligned}
        \right.
    \end{equation}
    где $A\in\BF{R}^{n\times n}$, имеет место следующая
    \begin{theor}[Критерий управляемости]
        Линейный объект $\eref{LIN_DS}$ управляем тогда и только тогда, когда выполнено условие 
        $$
            \rg K_{A,b} = n,
        $$
        где
        $$
            K_{A,b} = \left[b,\; Ab,\; \cdots,\; A^{n-1}b\right]\in\BF{R}^{n\times n}
        $$.
    \end{theor}

    \proof \cite[стр. 305]{voron2}
    \par
    \textbf{Программное управление}
    \par
    Из доказательства критерия управляемости было получено, что управление вида
    $$
        u(t) = b^Te^{A^T(t_1-t)}W^{-1}(t_1)x_1
    $$
    переводит систему за время $t_1$ в состояние $x_1$ из состояния $x_0 = 0$. Это пример {\it программного управления}, когда в системе нет обратной связи.

    Как перевести систему из произвольного состояния $x_0$ в $x_1$?

    Используя идею аналогичную идеи из доказательства, получим управление вида
    $$
        u(t) = b^Te^{A^T(t_1-t)}W^{-1}(t_1)(x_1 - e^{At_1}x_0).
    $$

\subsection{Модальное управление}

    \defi{\it Модальное управление} --- это управление, в котором цель управления достигается за счет назначения корней характеристического полинома ЗАМКНУТОЙ системы. Здесь под целью управления имеется в виду настройка начальной фазы работы САУ или, строго говоря, {\it переходного режима}. Позже будут рассмотрены различные характеристики переходного режима.

    Модальное управление {\it по состоянию} имеет вид
    $$
        u = -kx,
    $$
    где $k\in\BF{R}^{1\times n}$ --- вектор коэффициентов обратной связи. В результате замыкания получим замкнутую систему
    $$
        \left\{
        \begin{aligned}
            \dot x &= (A - bk)x,\\
            x(0) &= x_0.
        \end{aligned}
        \right.
    $$

    \begin{theor} 
        Пусть пара $\{A,\; b\}$ --- управляема и задан полином $\gamma(s)$. Тогда существует вектор 
        $k\in\BF{R}^{1\times n}$, 
        такой что 
        $\sigma(A-bk) = \left\{s_1,\; s_2,\;\ldots,\; s_n\right\}$, где $s_i$ --- корни $\gamma(s)$.
    \end{theor}

    \textbf{Алгоритм построения вектора $k$}

    \begin{enumerate}
    \item 
        Вычислить коэффициенты полинома $\gamma(s) = \prod_{i=1}^{n}(s-s_i)$;
    \item 
        Вычислить $K_{A,b}$ ($|K_{A,b}|\neq0$, если система управляема);
    \item 
        Найти $\chi_A(s) = s^n + a_{n-1}s^{n-1} + \ldots + a_1 s + a_0$;
    \item 
        Вычислить $K_{\widehat A,\; \widehat b}$, $\{\widehat A,\; \widehat b\}$ --- каноническая форма управляемости;
    \item 
        Перемножить $M = K_{\widehat A, \widehat b} K^{-1}_{A,b}$;
    \item 
        Вычесть $\widehat k_i = \gamma_{i-1} - a_{i-1}, i=\cnt{1,n}$;
    \item 
        Умножить: $k = \widehat k M$;
    \item 
        Записать $u = -k x$.
    \end{enumerate}

    \examp Синтезировать модальное управление $u = - k x, \; k = (k_1\; k_2)$ со спектром $\{-1,\; -2\}$ для системы
    $$
        \left\{
        \dot x =
        \begin{pmatrix}
            2 & -1\\
            1 & 3
        \end{pmatrix}
         x + \begin{pmatrix}1\\ -2\end{pmatrix}u\right..
    $$

    {\it Решение}
    $$
        \gamma(s)=(s+1)(s+2)=s^2+3s+2
    $$
    $$
        k_{A,B} = 
        \begin{bmatrix}
            1 & 4\\
            -2 & -5
        \end{bmatrix}
        ,
    $$
    так как $|k_{A,B}=3\neq0$, то система управляема
    $$
        X_{A(S)}=
        \begin{vmatrix}
            s\cdot I - A
        \end{vmatrix}
        =
        \begin{vmatrix}
            s-2 & -1\\
            1 & s-3
        \end{vmatrix}
        =(s-2)(s-3)+1 = s^2-5s+7
    $$
    $$
        \widehat A =
        \begin{pmatrix}
            0 & 1\\
            -2 & -3
        \end{pmatrix},\ 
        \widehat B=
        \begin{pmatrix}
            0 \\ 1
        \end{pmatrix},\ 
        k_{\widehat A,\widehat B} = 
        \begin{pmatrix}
            0 & 1\\
            -2 & -3
        \end{pmatrix}
    $$
    $$
        M = k_{\widehat A, \widehat B}\cdot k^{-1}_{A,B}
    $$
    $$
        k^{-1}_{A,B}=
        \begin{pmatrix}
            1 & 4\\
            -2 & -5
        \end{pmatrix}^{T}\cdot
        \frac{1}{|k_{A, B}|}
        =
        \begin{pmatrix}
            1 & -2\\
            4 & -5
        \end{pmatrix}\cdot
        \frac{1}{3}        
    $$
    $$
        M=\frac{1}{3}  
        \begin{pmatrix}
            0 & 1\\
            1 & -3
        \end{pmatrix}
        \begin{pmatrix}
            1 & -2\\
            4 & -5
        \end{pmatrix}
        =
        \begin{pmatrix}
            4 & -5\\
            -11 & 13
        \end{pmatrix}
        \frac{1}{3}  
    $$
    $$
        \widehat k_i=\gamma    _{i-1}-a_{i-1},\ i=\overline{1\dots n}
    $$
    $$
        \widehat k_1 = 2-7=5\ \widehat k_2 =3+5=8
    $$
    $$
        k=\widehat k\cdot M
    $$
    $$
        k=
        \begin{pmatrix}
        -5 & 8
        \end{pmatrix}
        \begin{pmatrix}
            4 & -5\\
            -11 & 13
        \end{pmatrix}
        \frac{1}{3}
        =
        \begin{pmatrix}
            -36 & 43
        \end{pmatrix}
    $$
    $$
        u -kx
    $$

    \textbf{Ответ: }
    $
        u=
        \begin{pmatrix}
            -36 & 43
        \end{pmatrix}
        x
    $

\section{Стабилизация по выходу. Наблюдаемость и наблюдатель.}

    \defi{Стабилизация по выходу} --- задача стабилизации, при которой необходимо стабилизировать данную систему, если известен измеряемый выход $y(t)$, зависящий от состояния $x(t)$. Система имеет вид:

    \begin{equation}\label{DS}
        \left\{
        \begin{aligned}
            \dot x &= Ax + bu,\\
            y &= c x
        \end{aligned}
        \right.
    \end{equation}

    \begin{defi}\label{DEF}
        Динамическая система  называется \textit{наблюдаемой}, если для любого $t_0$ из условия $y(t)\equiv 0$ при $t\ge t_0$ следует $x_0 = 0$ и, наоборот, из $x_0=0$ следует $y(t)\equiv0$ при $t\ge t_0$.
    \end{defi}

    Для линейной системы \eref{DS} при $u\equiv0$ Определение~\ref{DEF} эквивалентно следующему.
    
    \begin{defi}
        Система \eref{DS} называется \textit{наблюдаемой}, если для любых начальных условий $x_1\neq x_0$ следует, что $y(x_1, t)\not\equiv y(x_0, t)$.
    \end{defi}

    Для того чтобы различить по выходу $y(t)$ начальное состояние $x(t)$ необходимо ввести понятие наблюдателя.

    \defi{Наблюдатель} --- динамическая система, которая позволяет оценить (различить) по выходу $y(t)$ начальное состояние $x(t)$.

    Выходом наблюдателя является оценка $\tilde x(t)$ реального состояния системы $x(t)$, т.е. $|x(t)-\tilde x(t)|\rightarrow 0$ при $t\rightarrow\infty$.

    Наблюдатель можно строить в различных формах. Например, самая общая форма линейного наблюдателя имеет вид
    $$
    \dot{\tilde x} = H\tilde x + h_y y + h_u u,
    $$
    где $y$ и $u$ --- известные величины, а $\tilde x (t)$ --- оценка состояния системы.

\subsection{Критерий наблюдаемости для линейных систем.}

    \begin{theor}[Критерий наблюдаемости]
        Система \eref{DS} наблюдаема тогда и только тогда, когда $\rg N_{c,A} = n$, где
        $$
            N_{c,A} =
            \begin{pmatrix}
                c\\
                cA\\
                \vdots\\
                cA^{n-1}
            \end{pmatrix}
            \in\BF{R}^{n\times n}.
        $$
    \end{theor}
    
    \begin{proof}
    \cite[стр. 315-318]{voron2}
    \end{proof}

\subsection{Наблюдатель Люенбергера}\cite{andr}

    Так как матрицы $A,\; b$ и $c$ известны, то их можно использовать при построении наблюдателя. Например, наблюдатель

    $$
        \dot{\tilde x} = A\tilde x + bu
    $$

    дает оценку только для устойчивых систем. Запишем уравнение для ошибки $e(t) = \tilde x (t) - x(t)$, вычитая из уравнения наблюдателя уравнение объекта \eref{DS}:

    $$
        \dot e = A e.
    $$
    
    Из устойчивости матрицы $A$ получим $e(t)\rightarrow0$ при $t\rightarrow\infty$.

    Отсюда видно, что скорость сходимости ошибки к нулю определяется целиком исходным объектом.

    Если же система неустойчива, либо есть требование к скорости сходимости оценки, то рассмотренный наблюдатель не подходит.

    Классическим наблюдателем является наблюдатель Люенбергера вида
    
    $$
        \dot{\tilde x} = A\tilde x + l (y-c\tilde x) + b u,
    $$
    
    где $y = cx$, $l\in\BF{R}^{n\times 1}$ --- вектор параметров.

    Снова записав уравнение для ошибки, получим
    
    $$
        \dot e = A e - l\cdot c\cdot e = A_l e,
    $$
    
    где $A_l = A-l\cdot c$. Для последней матрицы возможен выбор произвольного желаемого спектра за счет выбора вектора $l$ (при условии, что исходная система наблюдаема).

\section{Теорема существования наблюдателя Люенбергера}

    \begin{theor}
        Пусть система \eref{DS} наблюдаема и задан спектр $\Gamma=\{s_1,\ldots, s_n\}$ --- корни полинома $\gamma(s)=s^n+\sum_{i=0}^{n-1}\gamma_i s^i$.
        Тогда существует $l\in\BF{R}^{n\times 1}$, такой что $\sigma(A-l\cdot c) = \Gamma$.
    \end{theor}

    \begin{statement}[Каноническая форма наблюдаемости]
        Для любой {\it наблюдаемой} системы \eref{DS} найдется матрица $M$, такая что $|M|\neq0$ и для $z=Mx$ выполнено
        $\dot z = \widehat A z + \widehat b u$,
        где
        $$
        \widehat A^T =
        \begin{pmatrix}
            0& 1& 0& \ldots& 0& 0\\
            0& 0& 1& \ldots& 0& 0\\
            \hdotsfor{6}\\
            -a_0& -a_1& -a_2 &\ldots& -a_{n-2}& -a_{n-1}
        \end{pmatrix}\quad
        \widehat c=
        \begin{pmatrix}
            0 & \ldots & 0 & 1
        \end{pmatrix}
        $$
    \end{statement}

    \begin{proof}
        Пусть
        \begin{align*}
            \begin{cases}
                \dot x &= Ax,\ A\in\mathbb{R}^{n\times n} \\
                y &= Cx,\ C\in\mathbb{R}^{n\times n}
            \end{cases}
        \end{align*}

        Из условия, что 
        $
            \forall x_1\neq x_2 \Rightarrow y(t_ix_1) \neq y(t_ix_2)
        $
        (где $y(t_ix_k)$ --- выход по $x_k$) мы може сделать вывод:
        $rang\ N_{C, A}=n$.\\
        По формуле Коши получаем, что 
        $
            \dot x = Ax \Rightarrow x(t) = Atx_0\ \text{и}\ y(t)= e^{At}x_0C
        $\\
        Докажем от противного:\\
        Пусть $rang\ N_{C, A}<n$ Тогда $\exists\ \vec v\neq0:\ N_{C,A}\cdot\vec v = 0$\\
        По теореме Гамильтона-Келли:
        \begin{equation}
            A^q=\sum_{i=0}^{n-1}\alpha_i^q\cdot A^i,\ q\ges 0\ \ C\cdot A^q\cdot v=\sum_{i=0}^{n-1}\alpha_i^q\cdot C \cdot A^i \cdot v = 0 
            \label{EQ21}
        \end{equation}

        Из \eref{EQ21} следует, что
        
        \begin{align*}
            C\cdot& v = 0 \\
            C\cdot& A \cdot v = 0 \\
            C\cdot& A^{n-1}\cdot v = 0
        \end{align*}


        Выберем $x_1=0$, $x_2=v$. Тогда
        \begin{equation}
            y(t,v)=C\cdot e^{At}\cdot x_0 = \sum_{i=0}^\infty\frac{C\cdot A^i\cdot v\cdot t^i}{i!}
            \label{EQ21_1}
        \end{equation}

        Из \eref{EQ21_1} следует, что
        \begin{equation*}
        \left.
            \begin{aligned}
                y(t,0)&=0 \\
                y(t,v)&=\sum_{i=0}^\infty\frac{C\cdot A^i\cdot v\cdot t^i}{i!}=0
            \end{aligned}
        \qquad \right|\Rightarrow rand\ N_\text{C, A}=n
        \end{equation*}

        Докажем в обратную сторону: если $rand\ N_{C, A}=n$, то система наблюдаема\\
        Пусть $x_1\neq x_2$ и $y(t_ix_1)=y(t_ix_2)$.
        Тогда $0=y(t_ix_1)-y(t_ix_2)=C\cdot e^{At}\cdot x_1-C\cdot e^{At}\cdot x_2=C\cdot e^{At}\cdot v$, где $v=x_1-x_2$\ и $v\neq 0$. Тогда $C\cdot e^{At} = 0\ \forall\ C\in [0; +\infty)$
    \end{proof}

    \textbf{Алгоритм построения вектора $l$}

    \begin{enumerate}
        \item Вычислить коэффициенты полинома $\gamma(s) = \prod_{i=1}^{n}(s-s_i)$;
        \item Вычислить $N_{c,A}$ ($|N_{c,A}|\neq0$, если система наблюдаема);
        \item Найти $\chi_A(s) = s^n + a_{n-1}s^{n-1} + \ldots + a_1 s + a_0$;
        \item Вычислить $N_{\widehat c, \widehat A}$, $\{\widehat c,\; \widehat A\}$ --- каноническая форма наблюдаемости;
        \item Перемножить $M^{-1} = N^{-1}_{c, A} N_{\widehat c, \widehat A}$;
        \item Вычесть $\widehat l_i = \gamma_{i-1} - a_{i-1}, \; i=\cnt{1,n}$;
        \item Умножить: $l =  M^{-1}\widehat l$;
        \item Записать $\dot{\tilde x} = A\tilde x + l(y - c\tilde x ) + b u$
    \end{enumerate}


    \examp
    {
        Синтезировать наблюдатель Люенбергера со спектром $\{-1,\; -2\}$ для системы
        $$
            \left\{
            \begin{aligned}
            \dot x &=
            \begin{pmatrix}
            2 & -1\\
            1 & 3
            \end{pmatrix}
             x + \begin{pmatrix}1\\ -2\end{pmatrix}u,\\
            y &= (1\;\; -2) x
            \end{aligned}
            \right..
        $$
    }

    \textit{Решение}

    \begin{enumerate}
        \item Вычислим полином $\gamma(s)$:
            $$\gamma(s)=(s+1)(s+2)=s^2+3s+2$$
        \item Вычислим $N_{C, A}$:\\
            $
                N_{C, A} =
                \begin{pmatrix}
                 1 & -2\\
                 0 & -7
                \end{pmatrix} 
            $
            так как $|N_{C, A}|=-7$ и $|N_{C, A}|\neq0$, то система наблюдаема
        \item Найдем 
        $
            \chi_A(C)=|SI-A|=
            \begin{vmatrix}
                s-2 & 1\\
                -1 & s-3
            \end{vmatrix}
            =(s-2)(s-3)+1=s^2-5s+7
        $
        \item Вычислим $N_{\widehat C,\widehat A}$\\
            Для этого вычислим $\widehat C:$\ \quad$\widehat C=(10)$
            И $\widehat A:$\\
            $
                \widehat A=
                \begin{pmatrix}
                    0 & -7\\
                    1 & 5
                \end{pmatrix}
                \Rightarrow
                N_{\widehat C,\widehat A} =
                \begin{pmatrix}
                    1 & 0\\
                    0 & -7
                \end{pmatrix}
            $ 
        \item Получим матрицу $M^{-1}$: $M^{-1}=N_{C, A}^{-1}\cdot N_{\widehat C, \widehat A}$\\
            $$
                M^{-1}=
                \begin{pmatrix}
                    -7 & 2\\
                    0 & 1
                \end{pmatrix}
                \begin{pmatrix}
                    1 & 0\\
                    0 & -7
                \end{pmatrix}
                \cdot
                \frac{1}{-7}=
                \begin{pmatrix}
                    1 & 2\\
                    0 & 1
                \end{pmatrix}
            $$
        \item Вычислим вектор $l$:
            \begin{align*}
                \widehat l_i &= p_{i-1} - \alpha_{i_1},\ i=\overline{1\dots n} \\
                \widehat l_1 &= p_{0} - \alpha_{0} = -5 \\
                \widehat l_2 &= p_{1} - \alpha_{1} = -8
            \end{align*}
        \item Вычислим $l$:
        $$
            l =
            \begin{pmatrix}
                1 & 2\\
                0 & 1
            \end{pmatrix}
            \begin{pmatrix}
                -5 \\
                -8
            \end{pmatrix}
            =
            \begin{pmatrix}
                -13 \\
                -8
            \end{pmatrix}
        $$
        \item Выпишем ответ:
        $$
            \dot{\widetilde x} =
            \begin{pmatrix}
                2 & -1 \\
                1 & 3
            \end{pmatrix}
            \cdot \widetilde x
            +
            \begin{pmatrix}
                -13 \\
                -8
            \end{pmatrix}
            \left(y-\begin{pmatrix}1 & -2\end{pmatrix}\widetilde x\right)
            +
            \begin{pmatrix}
                1 \\ -2
            \end{pmatrix}
            u
        $$
    \end{enumerate}

\section{Модальное управление по выходу. Принцип разделения задач стабилизации и наблюдения}

    После того, как была получена оценка состояния, можно надеяться, что управление вида
    \begin{equation}\label{CONTR}
        u = -k \tilde x,
    \end{equation}
    где $\tilde x (t)$ --- оценка состояния $x(t)$, будет обладать стабилизирующей способностью рассмотренного ранее модального управления по состоянию.

    Вопрос: решает ли задачу стабилизации управление \eref{CONTR}?

    Выпишем подробно замкнутую систему, которая получается в результате подстановки такого управления. Имеем
    $$
        \left\{
        \begin{aligned}
        \dot x = Ax+bu,\\
        y=cx,\\
        \dot{\tilde x} = A\tilde x + l(y-c\tilde x) + bu,\\
        u = -k\tilde x.
        \end{aligned}
        \right.
    $$
    После подстановок:
    $$
        \left\{
        \begin{aligned}
        \dot x = Ax-bk\tilde x,\\
        \dot{\tilde x} = lc x + (A- bk-lc)\tilde x .\\
        \end{aligned}
        \right.
    $$

    Последняя система удобно переписывается в виде
    $$
        \begin{pmatrix}
        \dot x\\
        \dot{\tilde x}
        \end{pmatrix}
        =
        \overbrace{\begin{pmatrix}
        A & -bk\\
        lc& A-bk-lc
        \end{pmatrix}}^{ \bar A}
        \begin{pmatrix}
        x\\
        \tilde x
        \end{pmatrix}.
    $$

    Нетрудно видеть, что характеристический полином замкнутой системы с наблюдателем равен $z(s) = |sI_{2n} - \bar A | = |sI_n - (A-bk)||sI_n-(A-lc)|$. Заметим, что $\deg z = 2n$.

    Это наблюдение заключает в себе {\it принцип разделения задач стабилизации и наблюдения} --- их можно выполнять независимо друг от друга.

    Весь алгоритм состоит из двух этапов, рассмотренных ранее: построение наблюдателя и построение модального управления по состоянию.

    Пусть требуется присвоить замкнутой системе характеристический полином $\gamma_1(s)$, используя наблюдатель Люенбергера с характеристическим полиномом $\gamma_2(s)$.

    \begin{enumerate}
        \item Построить для полинома $\gamma_1(s)$ модальное управление (найти вектор $k$);
        \item Построить наблюдатель Люенбергера для полинома $\gamma_2(s)$;
        \item Выписать наблюдатель и модальное управление  (это ответ).
    \end{enumerate}


\end{document}